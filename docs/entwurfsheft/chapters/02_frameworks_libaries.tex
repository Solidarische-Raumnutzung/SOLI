\chapter{Frameworks und Libaries}
\label{ch:frameworks_libaries}

Für das Produkt \textit{Soli} werden unterschiedliche Frameworks und Libraries verwendet, um die Entwicklung zu erleichtern und die Qualität des Codes zu erhöhen.

Im Backend wird das \gls{Spring Boot} Framework verwendet.
Zur Datenpersistenz wird \gls{PostgreSQL} über HikariCP mittels \gls{SpringData} JPA verwendet.
Datenbank Migrationen werden mit \gls{Flyway} durchgeführt.
Zum Verschicken von E-Mails wird Spring Boot Mail verwendet.
Die Authentifizierung und Sicherheit wird mit Spring Boot Security realisiert, der KIT-Login wird mit dem OAuth2-Client von Spring Boot realisiert.
Außerdem werden statische \gls{HTML}-Seiten mit der Java Template Engine (JTE) generiert.

Im Frontend wird das \gls{CSS}-Framework \gls{DaisyUI} in Kombination mit TailwindCSS zur Gestaltung der Oberfläche verwendet.
Zur Umsetzung eines gut integrierten und geprüften Kalenders wird die \gls{JavaScript}-Bibliothek \gls{FullCalendar} verwendet.

Zum Testen der Anwendung wird \gls{JUnit} eingesetzt.
Zum Mocken von Objekten wird \gls{Mockito} verwendet.

Als Werkzeug zur Build-Automatisierung wird \gls{Gradle} verwendet.

Zudem wird \gls{Docker} verwendet, um die Anwendung und ihre Abhängigkeiten in \gls{Container}n zu betreiben.
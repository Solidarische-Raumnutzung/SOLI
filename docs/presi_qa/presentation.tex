%% Beispiel-Präsentation mit LaTeX Beamer im KIT-Design
%% entsprechend den Gestaltungsrichtlinien vom 1. August 2020
%%
%% Siehe https://sdqweb.ipd.kit.edu/wiki/Dokumentvorlagen

%% Beispiel-Präsentation
\documentclass{sdqbeamer}
\usepackage{booktabs}
\usepackage{graphicx}
\usepackage{xcolor}
\usepackage{listings}
\usepackage{tikz}
\usepackage{amssymb}


\lstdefinelanguage{GitLog}{
  morekeywords={feat,fix,chore,ci,Merge,draft},
  sensitive=false,
  morecomment=[l]{\#}
}

\lstset{
  language=GitLog,
  basicstyle=\ttfamily,
  keywordstyle=\color{blue},
  commentstyle=\color{gray},
  breaklines=true,
  columns=fullflexible,
  frame=single,
  keepspaces=true,
}

\AtBeginSection[]{
  \begin{frame}
  \vfill
  \centering
  \begin{beamercolorbox}[]{title}
    \centering\usebeamerfont{title}\huge\underline\insertsectionhead\par%
  \end{beamercolorbox}
  \vfill
  \end{frame}
}


%% Titelbild
\titleimage{banner_2020_kit}

%% Gruppenlogo
%\grouplogo{mylogo}

%% Gruppenname und Breite (Standard: 50 mm)
\groupname{Human-Computer Interaction and Accessibility}
\groupnamewidth{50mm}
\grouplogo{grouplogo.jpg}

% Beginn der Präsentation

\title[Solidarische Raumnutzung Qualitätssicherung]{Kolloquium Qualitätssicherung zur Solidarischen Raumnutzung}
\subtitle{PSE-Projekt WS24/25}
\author[]{Alexander Klee, Jannik Hönlinger, Johannes Frohnmeyer, Ben Steinle und Antonia Ammon}

\date{\today}

% Literatur

\usepackage[citestyle=authoryear,bibstyle=numeric,hyperref,backend=biber]{biblatex}
\addbibresource{presentation.bib}
\bibhang1em

\begin{document}

%Titelseite
\KITtitleframe

%Inhaltsverzeichnis
\begin{frame}{Inhaltsverzeichnis}
\tableofcontents
\end{frame}

\section{Tests und Coverage}

\subsection{Unit Tests}
\begin{frame}{\insertsubsectionhead}
    \begin{columns}
        \column{.5\linewidth} \begin{itemize}
            \item 195 Unit und Integration Tests
            \item 76\% Code Coverage
        \end{itemize}
        \column{.5\linewidth} \begin{table}[h]
            \centering
            \renewcommand{\arraystretch}{1.3}
            \begin{tabular}{l|c}
                \textbf{Paket} & \textbf{Line Coverage} \\
                \hline
                \hline
                \textit{Controller}  & 69\% -> 77\% \\
                \textit{Domain}      & 87\% -> 91\%\\
                \textit{DTO}         & 79\% -> 83\%\\
                \textit{Filter}      & 100\% -> 90\% \\
                \textit{Repository}  & 100\% \\
                \textit{Service}     & 85\% -> 84\% \\
                \hline
                \textit{Gesamt}      & 67\% -> 76\% \\
            \end{tabular}
            \caption{Coverage der verschiedenen Pakete}
            \label{tab:progress}
        \end{table}
    \end{columns}
\end{frame}

\subsection{Technische Umsetzung}
\begin{frame}{\insertsubsectionhead}
    \begin{itemize}
        \item JUnit Jupiter 5
        \item Mockito für Controller Tests, sonst Integration
        \item Docker Compose für kontrollierte Umgebung
        \item Manuelle Korrekturen von JTE
    \end{itemize}
\end{frame}

\subsection{Szenarien}
\begin{frame}{\insertsubsectionhead}
    \begin{itemize}
        \item Kernfunktionalität (keine Edge Cases)
        \item Durch nicht-Entwickler durchgeführt
        \item Auch mit Screen-Reader
        \item 100\% Erfolgsquote
    \end{itemize}
\end{frame}



\section{Codequalität}

\subsection{Finale Werkzeuge}
\begin{frame}{\insertsubsectionhead}
    \begin{itemize}
        \item IntelliJ: Automatisierte Imports
        \item IntelliJ: Code-Formatierung (Indentierung, Leerzeilen)
        \item Manuelle Reviews
        \item CodeQL auf GitHub
    \end{itemize}
\end{frame}

\subsection{Andere Werkzeuge}
\begin{frame}{\insertsubsectionhead}
    \begin{itemize}
        \item Spotless
        \item CleanThat
        \item Google Java Format (auch AOSP)
        \item Palantir Java Format
        \item ImportOrder
    \end{itemize}
\end{frame}

\subsection{Fazit}
\begin{frame}{\insertsubsectionhead}
    \begin{itemize}
        \item IntelliJ genügt für die meisten Zwecke
        \item Auto-Formatter zu spät evaluiert
    \end{itemize}
\end{frame}



\section{Management}

\subsection{Ansatz}
\begin{frame}{\insertsubsectionhead}
    \begin{itemize}
        \item Issues für nontriviale Änderungen
        \item PRs für Code-Reviews
        \item Triviales (Typos) direkt
        \item 35 Issues, 140 Commits
        \item GitHub CI
    \end{itemize}
\end{frame}



\section{Beispiele}

\subsection{Mobile-UI}
\begin{frame}{\insertsubsectionhead}
    \begin{itemize}
        \item Z-Order
        \item Hamburger Menu
        \item FullCalendar
        \item Flexbox Footer
    \end{itemize}
\end{frame}

\subsection{Lesbarkeit}
\begin{frame}{\insertsubsectionhead}
    \begin{itemize}
        \item Farben
        \item Neue Symbole
        \item Neue Event-Badges
        \item Tooltips
        \item Konsistenz in Text und UI
        \item Redesign der Konfliktlösung
    \end{itemize}
\end{frame}



\section{Fragen?}

\end{document}

%!TEX root = ../main.tex

\chapter{Werkzeuge}
\label{ch:tools}

In diesem Kapitel werden die verschiedenen Werkzeuge beschrieben, die im Projekt verwendet wurden, um die Entwicklung, das Testen und die Qualitätssicherung zu unterstützen.

\section{JUnit}
JUnit ist ein Framework zum Schreiben und Ausführen von Unit-Tests in Java.
Es ermöglicht das Testen einzelner Methoden und Klassen, um sicherzustellen, dass sie wie erwartet funktionieren.
Wir setzen dieses Framework für Unittests auf allen Ebenen sowie für Integrationstests ein.
Damit können wir sicherstellen, dass der Code korrekt funktioniert und Fehler frühzeitig identifizieren.
Da bereits in der Implementierungsphase Unittests für die einzelnen Komponenten geschrieben wurden, konnten wir sicherstellen,
dass unsere Anpassungen in der Qualitätssicherung keine neuen Fehler eingeführt haben.
Außerdem haben wir weitere Tests geschrieben, um sicherzustellen, dass die neuen Funktionen korrekt funktionieren.

\section{JaCoCo}
JaCoCo (Java Code Coverage) ist ein Werkzeug zur Messung der Testabdeckung.
Es zeigt an, welche Teile des Codes durch Tests abgedeckt sind und welche nicht, was hilft, ungetesteten Code zu identifizieren.
Wir haben JaCoCo verwendet, um sicherzustellen, dass unsere Tests eine ausreichende Abdeckung des Codes haben.
Dadurch können wir Fehler durch ungetesteten Code vermeiden und die Qualität des Codes verbessern.

\section{GitHub CI}
GitHub CI (Continuous Integration) ist ein Dienst, der automatisch Builds und Tests ausführt, wenn Code in ein GitHub-Repository eingecheckt wird.
Dies hilft, sicherzustellen, dass der Code immer in einem funktionsfähigen Zustand ist.
Wir setzten GitHub CI sowohl bei jedem Commit als auch bei jedem Pull-Request ein,
um unsere Tests auszuführen, die Coverage zu überprüfen und Encoding-Probleme zu identifizieren.
Dadurch konnten wir den Review-Prozess beschleunigen und sicherstellen, dass der Code immer in einem funktionsfähigen Zustand ist.

\section{Security Tests}
GitHub bietet auch die Möglichkeit, automatische Sicherheitstests und Abhängigkeitsanalysen durchzuführen.
Mit diesen Tests können Sicherheitslücken und Schwachstellen in den Abhängigkeiten des Projekts identifiziert werden.
Außerdem nutzen wir Dependabot, um automatisch Abhängigkeiten zu aktualisieren und Sicherheitslücken zu schließen.

\section{Codequalität}
Zur Überprüfung und Verbesserung der Codequalität haben wir die Analysewerkzeuge von IntelliJ IDEA verwendet.
Diese Werkzeuge helfen, potenzielle Probleme im Code zu identifizieren und zu beheben, um die Qualität des Codes zu verbessern.
In Kombination mit ausführlichen Code-Reviews konnten wir sicherstellen, dass der Code so sauber und wartbar wie möglich ist.
Weitere Werkzeuge wie Checkstyle, Google Java Format, Palantir Java Format, Error Prone und andere
wurden als Optionen zur Automatisierung dieses Prozesses in Betracht gezogen und evaluiert,
zum Vermeiden des durch sie verursachten Mehraufwands durch aufwändige Konfiguration für bestenfalls minimale Verbesserungen aber nicht eingesetzt.

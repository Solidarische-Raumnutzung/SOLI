%------Ändern von Schriftschnitten - (Muss ganz am Anfang stehen !) ------------
\usepackage{fix-cm}


%------Umlaute -----------------------------------------------------------------
%   Umlaute/Sonderzeichen wie äüöß können direkt im Quelltext verwenden werden.
%    Erlaubt automatische Trennung von Worten mit Umlauten.
\usepackage[T1]{fontenc}
\usepackage[utf8]{inputenc}

%------Anpassung der Landessprache----------------------------------------------
\usepackage[ngerman]{babel}

%------Einfache Definition der Zeilenabstände und Seitenränder------------------
\usepackage{geometry}
\usepackage{setspace}

%------Schriftgrößenanpassung von einzelnen Textpassagen------------------------
\usepackage{relsize}

%------Trennlinien in Kopf- und Fusszeile
\usepackage[headsepline, footsepline, ilines]{scrlayer-scrpage}

%------Grafiken und Farben -----------------------------------------------------
\usepackage{graphicx}

%------Packet zum Sperren, Unterstreichen und Hervorheben von Texten------------
\usepackage{soul}

%------ergänzende Schriftart----------------------------------------------------
\usepackage{helvet}

%------Lange Tabellen-----------------------------------------------------------
\usepackage{longtable}
\usepackage{array}
\usepackage{ragged2e}
\usepackage{lscape}
\usepackage{tabularx}

%------DocTeX-------------------------------------------------------------------
\usepackage{changepage}
\usepackage{scalerel}
\newcolumntype{R}{>{\raggedright\arraybackslash}X}%

%------PDF-Optionen-------------------------------------------------------------
\usepackage[
  bookmarks,
  bookmarksopen=true,
  colorlinks=true,
  linkcolor=black,        % einfache interne Verknüpfungen
  anchorcolor=black,      % Ankertext
  citecolor=black,        % Verweise auf Literaturverzeichniseinträge im Text
  filecolor=black,        % Verknüpfungen, die lokale Dateien öffnen
  menucolor=black,        % Acrobat-Menüpunkte
  urlcolor=black,         % Farbe für URL-Links
  backref,                % Zurücktext nach jedem Bibliografie-Eintrag als
                          % Liste von Überschriftsnummern
  pagebackref,            % Zurücktext nach jedem Bibliografie-Eintrag als
                          % Liste von Seitenzahlen
  plainpages=false,       % zur korrekten Erstellung der Bookmarks
  pdfpagelabels,          % zur korrekten Erstellung der Bookmarks
  hypertexnames=false,
  colorlinks=true,   % zur korrekten Erstellung der Bookmarks
  linktocpage             % Seitenzahlen anstatt Text im Inhaltsverzeichnis verlinken
  ]{hyperref}
  \hypersetup{
    colorlinks=true, %set true if you want colored links
    linktoc=all,     %set to all if you want both sections and subsections linked
  }
  \usepackage{verbatim}
  \usepackage{lipsum}                     % Dummytext
  \usepackage{xargs}                      % Use more than one optional parameter in a new commands
  \usepackage[pdftex,dvipsnames]{xcolor}  % Coloured text etc.

  \usepackage[colorinlistoftodos,prependcaption,textsize=tiny]{todonotes}

%------Glossar------------------------------------------------------------------
\usepackage[translate=babel,toc]{glossaries}

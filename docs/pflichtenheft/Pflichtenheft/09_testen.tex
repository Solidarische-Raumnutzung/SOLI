\chapter{Testfälle und Testeszzenarien}
\label{chap:test}
In diesem Kapitel definieren wir die Testfälle und Testfallszenarien.

\section{Testfälle}

\begin{table}[htbp]
  \centering
  \begin{tabularx}{\textwidth}{ l|X|l }
      \textbf{Nr.} & \textbf{Beschreibung} & \textbf{Funktion} \\ \hline\hline
      ⟨T10⟩ & Landeseite besuchen &\ref{F10}\\
      ⟨T20⟩ & Login &\ref{F20} \\
      ⟨T30⟩ & Abmelden &\ref{F30} \\
      ⟨T40⟩ & Reservieren &\ref{F40} \\
      ⟨T50⟩ & Löschen von Terminen durch Administratoren &\ref{F50} \\
      ⟨T60⟩ & Deaktivieren eines Kontos &\ref{F60} \\
      ⟨T70⟩ & Benachrichtigung bei freiem Raum &\ref{F70} \\
      ⟨T80⟩ & Anzeige des Raumstatus &\ref{F80} \\
      ⟨T90⟩ & Stornierung einer Reservierung &\ref{F90} \\
      ⟨T100⟩ & Login mit Adminkonto &\ref{F100} \\
      ⟨T110⟩ & Deaktivierung von Gastkonten &\ref{F110} \\
      ⟨T120⟩ & Öffnungszeiten einstellen &\ref{F120} \\
      ⟨T130⟩ & Terminkonfliktauflösung &\ref{F130} \\
  \end{tabularx}
  \caption{Überblick der Testfälle.}
  \label{tab:test_table}
\end{table}

\pagebreak

\section{Testfallszenarien}\label{sec:testfallszenarien}
\begin{scenario}{10}{Besuch der Landeseite und Anmeldung/Abmeldung}
  \item[Ziel:] Sicherstellen, dass Nutzende die Landeseite aufrufen und sich erfolgreich anmelden bzw.\ abmelden können.
  \begin{enumerate}
    \item Der Nutzende besucht die Landeseite ⟨T10⟩.
    \item Der Nutzende loggt sich ein ⟨T20⟩.
    \item Der Nutzende meldet sich ab ⟨T30⟩.
  \end{enumerate}
\end{scenario}

\begin{scenario}{20}{Reservierung eines Raums und Stornierung}
  \item[Ziel:] Überprüfen, ob Nutzende erfolgreich Termine reservieren können.
  \begin{enumerate}
    \item Der Nutzende besucht die Landeseite ⟨T10⟩.
    \item Der Nutzende meldet sich an ⟨T20⟩.
    \item Der Nutzende wählt einen verfügbaren Raum aus und reserviert diesen ⟨T40⟩.
    \item Buchen ausserhalb der Öffnungszeiten sollte hierbei fehlschlagen.
    \item Überprüfen, ob der Raumstatus korrekt angezeigt wird und sich ggf.\ geändert hat ⟨T80⟩.
    \item Der Nutzende storniert die Reservierung ⟨T90⟩.
    \item Andere Nutzende, welchen sich diesen Termin vorgemerkt hatten, werden benachrichtigt, dass der Raum nun Frei ist ⟨T70⟩.
  \end{enumerate}
\end{scenario}

\begin{scenario}{30}{Terminverwaltung durch das Adminkonto}
  \item[Ziel:] Testen der administrativen Funktionalitäten zum Löschen von Terminen.
  \begin{enumerate}
    \item Ein Admin besucht die Landeseite ⟨T10⟩.
    \item Der Admin meldet sich an ⟨T100⟩.
    \item Der Admin löscht bestehende Termine ⟨T50⟩.
    \item Der Admin meldet sich ab ⟨T30⟩.
  \end{enumerate}
\end{scenario}

\pagebreak

\begin{scenario}{40}{Kontoverwaltung durch das Adminkonto}
  \item[Ziel:] Testen der administrativen Funktionalitäten zum Deaktivieren von Gastkonten und einstellen der Öffnungszeiten.
  \begin{enumerate}
    \item Ein Admin besucht die Landeseite ⟨T10⟩.
    \item Der Admin meldet sich an ⟨T100⟩.
    \item Der Admin deaktiviert die Anmeldung von Gastkonten ⟨T110⟩.
    \item Der Admin ändert die Öffnungszeiten für einen bestimmten Wochentag ⟨T120⟩.
    \item Der Admin öffnet die Ansicht \textit{Kontoliste} und deaktiviert ein Konto ⟨T60⟩.
    \item Der Admin meldet sich ab ⟨T30⟩.
    \item Das Anmelden mit einem Gastkonto sollte jetzt fehlschlagen ⟨T20⟩.
    \item Das Anmelden mit dem deaktiviertem Konto sollte ebenfalls fehlschlagen ⟨T20⟩.
    \item Die Ansicht \textit{Kalender} sollte die neuen Öffnungszeiten darstellen.
  \end{enumerate}
\end{scenario}

\begin{scenario}{50}{Terminkonflikt auflösung}
  \item[Ziel:] Überprüfen, ob ein Terminkonflikt richtig aufgelöst wird.
  \begin{enumerate}
    \item Konto 1 meldet sich an ⟨T20⟩.
    \item Konto 1 erstellt einen Termin, und gibt die zu testende Priorität, Raumteilungsoption und Zeitperiode ein ⟨T20⟩.
    \item Konto 1 meldet sich ab und Konto 2 meldet sich an.
    \item Konto 2 erstellt einen Termin, welcher den anderen überlappt.
    \item Es wird überprüft, ob die erwartete Konfliktauflösung stattgefunden hat und ggf.\ die dafür benötigten E-Mails versendet wurden ⟨T130⟩.
  \end{enumerate}
\end{scenario}

%!TEX root = ../presentation.tex
\section{Einleitung}

\begin{frame}{}{}
    EINLEITUNG HIER
\end{frame}

\subsection[nötige Funktionen]{Wasfür Funktionen sollen gewährleistet sein?}

\begin{frame}{Funktionen}
    \begin{itemize}
        \item intuitiv wie Kalender, aber \textrightarrow \space \textbf{Konfliktlösung} /*--> 3 Prio damit übersichtlich, höhere überschreibt niedrigere*/
        \item spontan nutzbar /*--> Statusbanner + schnell Buchen + mobile Endgeräte*/
        \item Raum teilen /*bei Buchung angeben, dass man teilen möchte, auch auf anfrage, dann mail*/
        \item  möglichst leicht les-/bedienbar für alle /*Darkmode, kein ausschließliches colorcoding*/
    \end{itemize}
\end{frame}

\subsection[Werdegang und Schwierigkeiten]{Wie sind wir dahin gekommen und was waren Schwierigkeiten?}

\begin{frame}{Wie sind wir dahin gekommen?}
    \begin{itemize}
        \item Befragungen
        \item Umsetzung diskutiert
    \end{itemize}
\end{frame}

\begin{frame}
    \begin{figure}
        \centering
        \includegraphics[width=0.6\linewidth]{pictures/BrainstormTafelbild}
        \label{fig: Tafelbild Brainstorm}
    \end{figure}
\end{frame}

\begin{frame}
    \begin{figure}
        \centering
        \includegraphics[width=0.6\linewidth]{pictures/BenutzeroberflächeKalender}
        \label{fig: Mockup Übersicht}
    \end{figure}
\end{frame}

\begin{frame}{Schwierigkeiten}
    \begin{itemize}
        \item Wie Konfliktlösung?
        \item Wie wird informiert oder verhandelt?
        \item Ist Hardware möglich?
    \end{itemize}
\end{frame}
